\chapter{Introduction}
\label{sec:introduction}
Most micro aerial vehicles (MAV) are flown on open terrain with only a few or none obstacle nearby. Flying close to a building or in a corridor is difficult. An inexperienced user can easily crash a MAV, which not only damages the vehicle but moreover can be hazardous for people.  An autonomous avoidance system can assist the pilot and prevent crashes, thus allowing MAVs to be piloted in obstacle rich environments. A reliable avoidance system open ups new field in which MAVs could be used, such as inspection of buildings.\\
Simplifying the navigation of MAVs and thereby increasing the safety is a major challenge for researchers. Dangerous situation should autonomously be detected and avoided to prevent crashes. Much research focuses on vision-based systems and digital image processing for autonomous obstacle avoidance \cite{heng2011autonomous} \cite{frew2004vision} \cite{ahrens2009vision}. These system are reliable, accurate, and can often simultaneously map the environment or be used for navigation. Unfortunately, a lot computational resources for data processing is required. This increases the power consumption and the size of the MAV as well as the price of the whole system. This constraints autonomous obstacle avoidance system to bigger and costly platforms. In some cases a simpler approach suffices. Mapping the environment or additional navigation is not always required but having an autonomous avoidance system is still beneficial for safety reasons. In \cite{gupta2015obstacle} the authors use ultrasonic sensors for autonomous obstacle avoidance. Their avoidance system has the advantage of being lightweight and having a low power consumption compared to vision-based systems.\\
The overall goal of this report is to increase the security during MAV flights with an autonomous obstacle avoidance system. We propose a system which is lightweight, has a low power consumption and requires only little computational resources. This allows our system to be implemented in small and low-cost platform.\\ 
Our obstacle avoidance system is based on four low-cost time-of-flight (ToF) infrared radiation (IR) sensors. The sensors are lightweight and have only a small energy consumption. Unfortunately, using only four sensors limits our system to the horizontal plane and reduces the observable range. However, the system can easily be extended with additional sensors.\\
Our avoidance system has to run in real time and immediately react to an obstacle in its proximity. A small delay could already lead to a crash. Therefore, we have implemented the avoidance system on a low level attitude controller. This allows to interfere immediately in the controller if a obstacle is approached and to avoid it.\\
For the avoidance we use an, in the autopilot computed, artificial potential field to calculate the repulsive force acting on our platform. The potential field is computed in real time with the distance measurements from the sensors. The repulsive force increases exponentially as the obstacle is approached. Using an potential field has the benefit of being easy to implement and computationally cheap.\\
Our proposed system should be appliable to different platforms as easy as possible. Hence, we decided to use a \textit{pixhawk} autopilot. The pixhawk can be used for different platforms. We simply have to flash the pixhawk with our firmware.\\
In this report we present the prove of concept of our proposed system. The system is able to autonomously detect obstacles, compute a repulsive force and push itself away. However, the system still needs tuning. The repulsive force is aggressive and might destabilise the MAV. Additionally, ranging sensor should be added to increase observable range of the system.
