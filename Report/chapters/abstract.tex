\chapter*{Abstract}
\addcontentsline{toc}{chapter}{Abstract}
Flying a MAV is relative easy on open terrain but flying the MAV close to or inside a building drastically increases the difficulty and can easily lead to a crash. With an assisted navigation system we can increase the security of MAV flight in a obstacle rich environment. Many solution to this problem already exist but they rely often on vision based system which are expensive and need a lot of the computational resources. In this thesis we present an autonomous avoidance system for MAVs that requires only four infrared time-of-flight ranging sensors. The sensors are lightweight and have low energy requirements. We have evaluated the sensor on their potential use for our autonomous obstacle avoidance systems. For the autopilot we use a pixhawk. This allows us to easily implement our avoidance systems on different MAVs. An artificial potential field is calculated with the ranging data and repels the platform away an obstacle. For the testing we piloted our system towards a wall and the artificial potential field successfully repelled the system.