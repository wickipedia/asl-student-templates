\chapter{Discussion}

\section{VL53L0X}
As a first part of the thesis, we had to analyse the VL53L0X on its potential use for our avoidance system. It is required that the sensor recognise different surfaces at different angles. In \cref{sec:vl53l0x} we have proven the sensors ability to fulfil these requirements, and we did not observe a significant trend. As a second test, we have tested the sensors performance, measuring at different distances. Increasing the distance trends to increases the sensors variance and to decrease its accuracy. This trend is small, therefore it is negligible. The mentioned experiments were all conducted inside, i.e, without a IR source. This indicates the avoidance systems potential use inside. It is possible to detect different obstacles, consisting of different material and at different angles. As a result, the sensor can be used for the ranging measurement of our proposed avoidance system if no IR source is interfering the sensor's measurements. \\
A IR source interfering the sensor's measurement has two effect. First, the data's variance increases, and, second, the sensor might fail to conduct a measurement. In case of direct sunlight, around 20\% of the measurements succeeded. This leads to a update rate of \unit[1]{Hz}, which is too slow and renders the data unusable for our avoidance system. We were able to solve this problem with a filter (\cref{eq:filter}). The filter we have used is only applicable for a static system, i.e., constant distance. If the drone were to be used outside, the filter needs to be extended to a dynamic system. This could be achieved by including the IMU's measurement in the a priori state estimate of the filter. \\
To conclude, the sensors have proven their potential use for our avoidance system. Without a IR source we do not have to filter the data and can directly process them. In the case of having a IR source interfering the measurements, a filter has to be implemented before further processing the date.  

\section{Avoidance System}

The overall goal of this thesis was to increase the security during MAV flights. Our approach consisted to implement on a low level an obstacle avoidance system. We have opted for to use a potential field to computed the repulsive force and four low power ranging sensors for the distance measurement. In \cref{sec:avoidance} we describe our successful running autonomous avoidance system using the VL53L0X ranging sensor. \\
Approaching the obstacle, the repulsive force computed by the potential field should exponentially increase. We successfully demonstrated in (\cref{fig:pf}) this behaviour. During the test it became apparent, when the manoeuvre is too aggressive the system might become unstable and the user has to counter steer. Further tuning of the potential field might solve this problem. Another option is too add a damping term in the potential field which stabilizes the system. As a downside, damping the system might lead in a too weak repulsive force, thus, increasing the chance of a crash.\\
The potential field's update rate of \unit[5]{Hz} sufficed for our test. In case of a rapid manoeuvre the rate might be too slow and thus the potential field might not be able to prevent a crash. We have implemented the potential field in a separate module instead of directly in the attitude controller. Computing the repulsive force directly in the controller increases the update rate of the potential field but on the same time it is dangerous since we change the low level controller. This could lead to undesired behaviour. \\
We evaluated the avoidance system with the VL53L0X's range limited to \unit[1.2]{m}. Changing the operation mode of the sensors to long distance increases the range to \unit[2]{m}, thus, the potential field could repulse the platform from the obstacle earlier and minimising the chance of a crash. Additionally, we could decrease gain of the potential field leading to a more subtle change in repulsive force and thereby increase stability. Unfortunately, changing the operation mode of the VL53L0X sensors decreases the accuracy and increases the sensitivity IR disturbance. Consequently, using the system outside limits the system's range to \unit[1.2]{m}.\\
We have tested the system only inside, i.e., without a IR source, and with only one obstacle. Further test need to be undertaken to investigate the system's behaviour with several obstacles or the operation outside. Potential fields show a inherent vulnerability to oscillation thus, flying with several obstacle or in a corridor might destabilise the system. In addition, not all obstacles might be detected since we only use fourn ranging sensors. Fortunately, the system can be equipped with additional ranging sensors. Using the system outside might lead to corrupted data due to the IR thus a filter similar to \cref{alg:filter} should be implemented.\\